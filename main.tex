%%
%% This is file `sample-authordraft.tex',
%% generated with the docstrip utility.
%%
%% The original source files were:
%%
%% samples.dtx  (with options: `authordraft')
%% 
%% IMPORTANT NOTICE:
%% 
%% For the copyright see the source file.
%% 
%% Any modified versions of this file must be renamed
%% with new filenames distinct from sample-authordraft.tex.
%% 
%% For distribution of the original source see the terms
%% for copying and modification in the file samples.dtx.
%% 
%% This generated file may be distributed as long as the
%% original source files, as listed above, are part of the
%% same distribution. (The sources need not necessarily be
%% in the same archive or directory.)
%%
%% The first command in your LaTeX source must be the \documentclass command.
\documentclass[sigconf]{acmart}
\usepackage[utf8]{inputenc}
\settopmatter{printacmref=false} % Removes citation information below abstract
\renewcommand\footnotetextcopyrightpermission[1]{} % removes footnote with conference information in first column
\pagestyle{plain} % removes running header

%%
%% \BibTeX command to typeset BibTeX logo in the docs
\AtBeginDocument{%
  \providecommand\BibTeX{{%
    \normalfont B\kern-0.5em{\scshape i\kern-0.25em b}\kern-0.8em\TeX}}}

%% Rights management information.  This information is sent to you
%% when you complete the rights form.  These commands have SAMPLE
%% values in them; it is your responsibility as an author to replace
%% the commands and values with those provided to you when you
%% complete the rights form.
%%\setcopyright{none}
%%\copyrightyear{2019}
%%\acmYear{2019}
%%\acmDOI{10.1145/1122445.1122456}

%% These commands are for a PROCEEDINGS abstract or paper.
%%\acmConference[Woodstock '18]{Woodstock '18: ACM Symposium on Neural
%%  Gaze Detection}{June 03--05, 2018}{Woodstock, NY}
%%\acmBooktitle{none}
%%\acmPrice{00.00}
%%\acmISBN{00}


%%
%% Submission ID.
%% Use this when submitting an article to a sponsored event. You'll
%% receive a unique submission ID from the organizers
%% of the event, and this ID should be used as the parameter to this command.
%%\acmSubmissionID{123-A56-BU3}

%%
%% The majority of ACM publications use numbered citations and
%% references.  The command \citestyle{authoryear} switches to the
%% "author year" style.
%%
%% If you are preparing content for an event
%% sponsored by ACM SIGGRAPH, you must use the "author year" style of
%% citations and references.
%% Uncommenting
%% the next command will enable that style.
%%\citestyle{acmauthoryear}

%%
%% end of the preamble, start of the body of the document source.
\begin{document}

%%
%% The "title" command has an optional parameter,
%% allowing the author to define a "short title" to be used in page headers.
\title{Scalability Issues in Cloud Computing and  Solution Approaches}

%% D.B.
\author{Daniel Bretschneider}
\email{ic19b035@technikum-wien.at}
\orcid{1234-5678-9012}
\affiliation{%
  \institution{University of Applied Sciences Technikum Wien}
  \streetaddress{Hoechstaedtplatz 6}
  \city{Vienna}
  \state{Austria}
  \postcode{1200}
}

%% F.D.
\author{Ferhat Dövme}
\email{ic15b046@technikum-wien.at}
\orcid{1234-5678-9012}
\affiliation{%
  \institution{University of Applied Sciences Technikum Wien}
  \streetaddress{Hoechstaedtplatz 6}
  \city{Vienna}
  \state{Austria}
  \postcode{1200}
}

%% the ben
\author{Behnam Ezazi}
\email{xyz@technikum-wien.at}
\orcid{1234-5678-9012}
\affiliation{%
  \institution{University of Applied Sciences Technikum Wien}
  \streetaddress{Hoechstaedtplatz 6}
  \city{Vienna}
  \state{Austria}
  \postcode{1200}
}

%% J.K
\author{julius Kosa}
\email{xyz@technikum-wien.at}
\orcid{1234-5678-9012}
\affiliation{%
  \institution{University of Applied Sciences Technikum Wien}
  \streetaddress{Hoechstaedtplatz 6}
  \city{Vienna}
  \state{Austria}
  \postcode{1200}
}


%%
%% By default, the full list of authors will be used in the page
%% headers. Often, this list is too long, and will overlap
%% other information printed in the page headers. This command allows
%% the author to define a more concise list
%% of authors' names for this purpose.
\renewcommand{\shortauthors}{}

%%
%% The abstract is a short summary of the work to be presented in the
%% article.
\begin{abstract}
Scalability is the fundamental attribute of every network,
system or infrastructure to increase or reduce its
performance, resources and functionalities in order to meet
the demands of a growing number of users and devices. High
scalability results in an optimization of the overall system
efficiency and cost-savings, while poor scalability eventuates
in poor system performance necessitating the replication of
system components, for example.\\

Cloud computing is a big shift from the traditional way
businesses think about IT resources, bringing several
benefits that encourage more and more organization to
outsource their services and data into the cloud. Another big
issue is the evolving sector of IoT with upcoming billions of
devices inter-connected via the Internet and will only be
possible due the paradigm of cloud computing.\\

When scaling a system or network, very different types of
problems can occur.This paper further contains several approaches 
on solving scalability related issues in cloud systems in order to
improve system performance.
\end{abstract}


%%
%% Keywords. The author(s) should pick words that accurately describe
%% the work being presented. Separate the keywords with commas.
\keywords{Scalability, Cloud, Cloud Computing, Issues, Problem
localization, horizontal scaling, vertical scaling, IoT}

%%
%% This command processes the author and affiliation and title
%% information and builds the first part of the formatted document.
\maketitle

%%
%% ===== SECTION: INTRODUCTION 
%%
\section{Introduction}
This section contains an introduction on cloud computing
in general and clarifies why scalability – among other
concepts – plays such an important role. Also provide examples
of scalable cloud-based systems (IaaS, PaaS, etc.) \\
Cloud computing is sharing software and hardware resources,
location independent, via the internet. Examples of cloud-based
systems are Server Scalability available by Infrastructure as a
Service (IaaS), Scaling of the Network with the need to scale
by consolidated data centers that host several VMs per physical
machine (often achieved by overprovisioning resources),
Scaling of the Platform by Platform as a Service (PaaS) offer
ready to use execution environments and convenient services
for applications.

\subsection{Related Work}
There are many papers related to scalability, cloud computing
and IoT, but there are only a few papers that combine all three
topics. Because of the growing devices connected to clouds
and the emerging IoT sector the paper is an attempt to provide
solutions to current problems.

%%
%% ===== SECTION: SCALABILITY 
%%
\section{Scalability - change!}
Scalability is the ability to provide sufficient performance
despite increasing demands. Grow or shrink, scaling is a change
of size and does not always mean increasing. Adjusting to
changing requirements is very important. Also declare what
does not concern to scalability (like replacing or something).
Scalability of a system can be measured along at least three
different dimensions [Neuman, 1994]. First, a system can be
scalable with respect to its size, meaning that we can easily
add more users and resources to the system. Second, a
geographically scalable system is one in which the users
and resources may lie far apart. Third, a system can be
administratively scalable, meaning that it can still be easy to
manage even if it spans many independent administrative
organizations. Unfortunately, a system that is scalable in one
or more of these dimensions often exhibits some loss of
performance as the system scales up.[3]

\subsection{Vertical Scalability}

What happens when a system is being scaled vertically? The
system is built up on different depending layers. User interface
layer, application layer and database layer are the typical layers
for the three-tier architecture. Each layer can be placed either
on the client machine or the server machine (cloud). Depending
on how the different layers are established we distinguish the
different kind of possible vertical scalability.

\subsection{Horizontal Scalability}

Horizontal scalability means to allocate the vertical
Scalability to different physical machines. Difficulties are the
requirements of certain services, which are not provided by
every physical machine and the requirements to latency.

\subsection{Comparison: Which is the better one?}

Discuss the both scaling methods based upon several
different factors. Those factors are: Pros/cons, when should
what method be used, how easy are they achieve, what
problems cloud possibly occur etc. \\

Introduce the concept of diagonally scaling.

\subsection{Diagonally Scalability}

Briefly explain how vertical and horizontal scalability can
be brought together to benefit from the advantages of both
methods.

\subsection{Diagonally Scalability}

Briefly explain how vertical and horizontal scalability can
be brought together to benefit from the advantages of both
methods.


%%
%% ===== SECTION: SCALABILITY ISSUES IN CLOUDS
%%
\section{Modifications}

Explain different issues concerning scalability in cloud
systems and define which items / components actually
interfere with scalability. Where should the overall focus
should be.

\subsection{Lack of Standardization}
Today there are many different standards...

\subsection{Volume}
Cloud scalability has to deal with various
volumes of users, resources and data involved in service
provision. Due the evolving IoT area, billions of devices will
be inter-connected by the year 2020[2].

\subsection{Lack of Ensuring autonomous scalability service management.[3]}
Whatever.

%%
%% ===== SECTION: IMPROVING SCALABILITY IN CLOUD SYSTEMS
%%
\section{IMPROVING SCALABILITY IN CLOUD SYSTEMS}
Now that the reader knows that several scalability related
problems exist in cloud environments, this section will inform
him about concepts and methods of solving these problems.

\subsection{Establish an international standard to support scalability in between clouds.}
blah. bluh. blih.

\subsection{IoT-Centric Cloud approach}
Cloud Computing is not only sharing the resources but also
maximizing the resources location independent. Virtualization
of physical devices in cloud based IoT to share the devices and
bring IoT functionalities into the cloud. Distribution over
heterogeneous platforms, spanning multiple management
domains. The ecosystem consists of local clouds and a global
cloud for real time big data and analytics.
A local Cloud can be created on-demand and provides service
to users in that geographical area. It can involve a large
number of nodes (sensors, actuators, smartphones, etc.)
The global Cloud is the “backbone-infrastructure” and
increases business opportunities for service providers. It
increases and provides a more dynamic resource management
and orchestration techniques, dynamically offloading from
clients/hosts to cloud. It provides a reliable real-time
communication from objects to applications and all of that
executed across borders.[1]

%%
%% IMAGE
%%

\subsection{Service Scalability assuring Process}
The approach is to establish an autonomously managing
process:

\begin{enumerate}

\item Quality metrics for measuring services are defined
(e.g.: throughput, - efficiency of handling service
invocations within a given time)

\item Acquire raw data items from monitored services

\item Compute scalability metrics. In case the metrics
reveal an acceptable scalability level, the control
goes back to step 2 and repeat steps 2 and 3. If the
metrics show a need to take actions steps 4,5 and 6
are performed.

\item Devise a remedy plan for enhancing suffered
scalability based on the current states of monitored
service. (scalability assuring schemes)

\item Run the selected scalability assuring schemes
according to the plan. Many, if not all, of the schemes
should be able to run without human administrators.

\item Analyze the result of applying the remedy plan and
learn from the whole process of enhancing
scalability. Making the whole scalability framework
more intelligent.
\end{enumerate}

Many Scalability assuring schemes are hardware-based
solutions. But there are also software-oriented schemes. Such
as Service Replication and Service Migration.[3]


\subsection{Benefits of Cloud Scalability}
Regarding performance, cost-efficiency etc.

%%
%% ===== SECTION: Conclusion
%%
\section{Conclusion}

As the section title says this will be the conclusion. Here will
summarize our “findings” and explain how cloud systems
should be scaled in order to prevent different types of
problems to even come up.	


%%
%% ===== SECTION: Citation and Bibliographies
%%
%% NOTE: Please don't delete. That's the way we have to
%% do the citations!
%%
\section{Citations and Bibliographies}

The use of \BibTeX\ for the preparation and formatting of one's
references is strongly recommended. Authors' names should be complete
--- use full first names (``Donald E. Knuth'') not initials
(``D. E. Knuth'') --- and the salient identifying features of a
reference should be included: title, year, volume, number, pages,
article DOI, etc.

The bibliography is included in your source document with these two
commands, placed just before the \verb|\end{document}| command:
\begin{verbatim}
  \bibliographystyle{ACM-Reference-Format}
  \bibliography{bibfile}
\end{verbatim}
where ``\verb|bibfile|'' is the name, without the ``\verb|.bib|''
suffix, of the \BibTeX\ file.

Citations and references are numbered by default. A small number of
ACM publications have citations and references formatted in the
``author year'' style; for these exceptions, please include this
command in the {\bfseries preamble} (before
``\verb|\begin{document}|'') of your \LaTeX\ source:
\begin{verbatim}
  \citestyle{acmauthoryear}
\end{verbatim}

  Some examples.  A paginated journal article \cite{Abril07}, an
  enumerated journal article \cite{Cohen07}, a reference to an entire
  issue \cite{JCohen96}, a monograph (whole book) \cite{Kosiur01}, a
  monograph/whole book in a series (see 2a in spec. document)
  \cite{Harel79}, a divisible-book such as an anthology or compilation
  \cite{Editor00} followed by the same example, however we only output
  the series if the volume number is given \cite{Editor00a} (so
  Editor00a's series should NOT be present since it has no vol. no.),
  a chapter in a divisible book \cite{Spector90}, a chapter in a
  divisible book in a series \cite{Douglass98}, a multi-volume work as
  book \cite{Knuth97}, an article in a proceedings (of a conference,
  symposium, workshop for example) (paginated proceedings article)
  \cite{Andler79}, a proceedings article with all possible elements
  \cite{Smith10}, an example of an enumerated proceedings article
  \cite{VanGundy07}, an informally published work \cite{Harel78}, a
  doctoral dissertation \cite{Clarkson85}, a master's thesis:
  \cite{anisi03}, an online document / world wide web resource
  \cite{Thornburg01, Ablamowicz07, Poker06}, a video game (Case 1)
  \cite{Obama08} and (Case 2) \cite{Novak03} and \cite{Lee05} and
  (Case 3) a patent \cite{JoeScientist001}, work accepted for
  publication \cite{rous08}, 'YYYYb'-test for prolific author
  \cite{SaeediMEJ10} and \cite{SaeediJETC10}. Other cites might
  contain 'duplicate' DOI and URLs (some SIAM articles)
  \cite{Kirschmer:2010:AEI:1958016.1958018}. Boris / Barbara Beeton:
  multi-volume works as books \cite{MR781536} and \cite{MR781537}. A
  couple of citations with DOIs:
  \cite{2004:ITE:1009386.1010128,Kirschmer:2010:AEI:1958016.1958018}. Online
  citations: \cite{TUGInstmem, Thornburg01, CTANacmart}. Artifacts:
  \cite{R} and \cite{UMassCitations}.

\section{Appendices}

Here comes the appandix.


%%
%% The next two lines define the bibliography style to be used, and
%% the bibliography file.
\bibliographystyle{ACM-Reference-Format}
\bibliography{sample-base}

%%
%% If your work has an appendix, this is the place to put it.
\appendix

\end{document}
\endinput
%%
%% End of file `sample-authordraft.tex'.
